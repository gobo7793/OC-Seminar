% !TEX root = main.tex

\section{Eindimensionale Repräsentation}
\label{sec:EindimRep}

Unter eindimensionaler Repräsentation wird vor allem die Darstellung der Daten in den elementaren Datentypen verstanden. Die Daten liegen dabei nur linear in einer bestimmten Reihenfolge vor, wodurch die Handhabung mit den Daten auch entsprechend einfach ist und in vielen Anwendungsfällen ohne großen Aufwand durchgeführt werden kann und entsprechend oft genutzt wird.

Zuerst wird im Abschnitt \ref{sec:BinCod} ein Überblick über Anwendungsfälle und mög\-liche COs für die Binär-Codierung gegeben. Anschließend wird in Abschnitt \ref{sec:IntCod} auf ganzzahlige Codierung allgemein bzw. auf Permutationen im Speziellen eingegangen, in Abschnitt \ref{sec:FloatCod} folgen Fließkomma-Darstellungen sowie in \ref{sec:StrCod} Zeichenketten.

\subsection{Binäre Codierung}
\label{sec:BinCod}

	Eine Codierung als Binärwerte bedeutet, dass die Werte, die vom GA bearbeitet werden, als eine Kette von 0 und 1 dargestellt sind. Der große Vorteil einer binären Codierung liegt vor allem darin, dass die Handhabung entsprechender Daten sehr einfach und platzsparend ist, weshalb diese Art der Darstellung prinzipiell von jeder Anwendung genutzt werden kann. Ebenfalls ein großer Vorteil liegt darin, dass entsprechende GAs aufgrund des geringen Alphabets sehr schnell und effizient sind.\cite{TacklingRealCodedGA}
	
	Vor allem folgende Arten von Anwendungen sind besonders dafür geeignet, mit binärer Codierung zu Arbeiten:\cite{Survey}
	\begin{description}
		\item[Classification Problem] Lösung soll in verschiedene Kategorien klassifiziert werden. \cite{NearestNeighborClassifier}
		\item[Multimodal Spin Lattice Problem] Suchen eines minimalen Ener\-gie\-zu\-stan\-des für 450 Spins mit je vier Zuständen auf einem 2D-Gitter. \cite{SelectionSchemesSpatialIsolation}
	\end{description}
	
	Passende Crossover-Operationen hierfür sind:
	\begin{description}
		\item[Self-Crossover] Tauschen von einzelnen Bits innerhalb des Chromosoms. \cite{SelfCrossover}
		\item[Supplementary Crossover] Nutzt das \textit{Center of Gravity}-Paradigma um Kin\-der zu erzeugen. \cite{SupplementaryCrossover}
		\item[Generalized crossover] Interpretiert Chromosom als Ganzzahl und dividiert diese durch eine andere, zufällig ausgewählte Ganzzahl. \cite{GeneralizedCrossover}
	\end{description}

\subsection{Codierung als Ganzzahlen und Permutationen}
\label{sec:IntCod}

	Da ganzzahlige Werte ebenfalls sehr einfach als Binärwerte dargestellt werden können, können COs für Binärwerte prinzipiell auch für ganzzahlige Werte eingesetzt werden. Natürlich gibt es auch entsprechende Anwendungsfälle und COs, die speziell für ganzzahlige Darstellungen geeignet und optimiert sind, auf die hier aber verzichtet wird und stattdessen auf Ketten von ganzzahligen Werten, also Permutationen, eingegangen wird.
	
	Der Unterschied zwischen einfachen, ganzzahligen Darstellungen und Permutationen liegt darin, dass bei Permutationen die komplette Zahlenfolge als mehrere aneinandergereihte Zahlen betrachtet werden muss. Entsprechend kön\-nen nicht einfache COs für Binärwerte und einfache, ganzzahlige Werte eingesetzt werden, sondern COs, welche die Zahlenketten entsprechend berücksichtigen. Dies ist \uA bei folgenden Anwendungen und Problemen der Fall:
	\begin{description}
		\item[Traveling Salesman Problem (TSP)] Mehrere Orte, die mit einer möglichst kurzen Strecke miteinander verbunden werden müssen. \cite{GAforTSP}
		\item[Graph Coloring Problem] Knoten eines Graphen mit möglichst wenig und verschiedenen Farben einfärben. \cite{OrderBasedForGCP}
		\item[Quadratic Assignment Problem] Summe der Distanzen zwischen Punkten minimieren, ähnlich wie TSP, nur mit quadratischer Kostenfunktion. \cite{COforQAP}
	\end{description}
	
	Passende COs für Permutationen lassen sich in verschiedene Kategorien einteilen, welche sich basierend auf dem grundlegenden Vorgehen der COs klassifizieren lassen. Im Folgenden werden nun einige positionsbasierte, kantenbasierte und folgenbasierte COs vorgestellt. Daneben gibt es aber auch noch weitere, wie \zB teilmengenbasierte, die auf Basis von Genom-Teilmengen arbeiten, \textit{Cut and Splice}-basierte, die einen Teil der Elterngenome tauschen, oder distanzbasierte COs, bei denen die Anzahl der unterschiedlichen Gene zwischen allen Elternteile gleich sein muss.
	
	\subsubsection{Positionsbasierte COs}
	
		beeinflussen die Gene ihrer Kinder basierend auf den Positionen der einzelnen Gene. Dies sind \uA folgende COs:
		
		\begin{description}
			\item[Partially Mapped Crossover (PMX)] Wählt zufällig zwei Punkte aus und tauscht alle Werte dazwischen mit dem anderen Elternteil. Bei der Er\-wei\-ter\-ung Uniform PMX werden einzelne Gene anstatt eines ganzen Segments getauscht. \cite{GAforTSP}\cite{COforPermutations}\cite{COforQAP}
			\item[Position Based Crossover (POS)] Wählt einige zufällige Positionen im ersten Elternteil aus und verschiebt die ausgewählten Werte zu den korrespondierenden Positionen im anderen Elternteil. \cite{COforPermutations}
		\end{description}
		
	\subsubsection{Kantenbasierte COs}
	
		erzeugen basierend auf Kanten der Eltern(-knoten) neue Kinder. Darunter fallen \uA folgende COs:
		
		\begin{description}
			\item[Edge Crossover] Wählt einen Knoten mit der geringsten Kantenzahl aus und fügt ihn zum neuen Kind hinzu und löscht ihn aus den anderen Kantenlisten. \cite{COforPermutations}
			\item[Edge Exchange Crossover (EXX)] Vererbt zunächst alle Elternkanten um einen Kreis zu bilden, entfernt dann aber ungültige Kanten. Falls kein Kreis gebildet werden konnte, werden alle Kanten vererbt. \cite{EdgeCOforTSP}
		\end{description}
	
	\subsubsection{Folgenbasierte COs}
	
		beeinflussen vor allem die Reihenfolge der Gene. Sie lassen sich in weitere Unterkategorien einteilen. Darunter fallen \zB folgende COs:
		
		\begin{description}
			\item[Merging Crossover (MOX)] Beide Eltern werden zuerst zufällig zu\-sam\-men\-ge\-fügt und anschließend in die beiden Kinder geteilt. \cite{OrderBasedForGCP}
			\item[Non-Wrapping Ordered Crossover] Erstellt Lücken und füllt diese wieder auf, ohne dabei die absolute Reihenfolge der Gene zu verlieren. \cite{GAforTSP}
		\end{description}
	
		Daneben gibt es auch noch weitere COs, welche sich neben den hier vorgestellten Unterkategorien für folgenbasierte COs (verschmelzende und absolute Reihenfolge) in \zB  sor\-tier\-ende oder angrenzende, folgenbasierte COs einsortieren lassen.


\subsection{Codierung als Fließkommazahl}
\label{sec:FloatCod}

	Im Gegensatz zur Codierung als Ganzzahlen bzw. Binärwerte lassen sich Daten, welche als Fließkommazahlen codiert sind, viel genauer Darstellen, haben da\-durch aber auch einen deutlich größeren Suchraum für mögliche Lösungen. Dennoch liegt gerade darin der Vorteil, da die Art der Datencodierung möglichst mit der Codierung des Suchraumes übereinstimmen sollte. Dadurch entfällt die notwendige Konvertierung der Daten, wodurch letztlich die Geschwindigkeit und damit die Effizienz des GAs deutlich erhöht wird. Ebenso ist von Vorteil, dass große Suchräume mit kontinuierlichen Daten auf einfache Art und Weise abgesucht werden können anstatt lediglich mit diskreten Daten wie bei der Binärcodierung. \cite{TacklingRealCodedGA}
	
	Entsprechend lassen sich vor allem Anwendungen bzw. Probleme, welche einen kontinuierlichen Suchraum besitzen, mit Fließkomma-Codierungen effizient ausführen. Darunter fallen \zB folgende Anwendungen und Probleme:
	
	\begin{description}
		\item[Animal Diet Formulation Problem] Erstellung eines Ernährungsplans mit möglichst vielen Nährstoffen zu minimalen Kosten. \cite{ConceptOfCOInRealCoded}
		\item[Electromagnetic Optimization] Handhabung und Optimierung von elektromagnetischen Werten in einem kontinuierlichem Suchraum. \cite{ElectromagneticRealEncoding}
		\item[Revenue Management in Airlines] Zuweisen von begrenzten Ressourcen ei\-ner Airline (Flugzeuge, Personal, ...), um den Gewinn zu maximieren. \cite{AirlineRevenueManagement}
	\end{description}
	
	Als Fließkommazahlen codierte Daten kann man auf mehrere Arten verarbeiten, mit deren Basis die COs funktionieren. Dies sind \zB folgende Arten bzw. COs:
	
	\begin{description}
		\item[Taguchi Crossover (TC)] Nutzt statistische Daten innerhalb einer Matrix um das beste Ergebnis zu finden. \cite{TaguchiCrossover}
		\item[Average Crossover] Modifizierter \textit{One-Point-Crossover}, bei dem die Durchschnittswerte zwischen beiden Eltern anstatt den Elternwerten der einzelnen Gene genutzt werden. \cite{ConceptOfCOInRealCoded}
		\item[Fuzzy Arithmetic Weighted Mean (FAWM)] Berechnet mithilfe der Fuzzy-Arithmetik die Werte der Kinder. \cite{AirlineRevenueManagement}
	\end{description}
	
	Daneben gibt es aber auch noch einige weitere Arten wie \zB den gewichteten Durchschnitt oder dem \textit{Center of Mass Crossover (CMX)}, welcher den gleichnamigen Ansatz nutzt um mithilfe des Massenzentrums der Eltern neue Kinder zu erzeugen. \cite{MultiParentRecombination}

\subsection{Codierung als Zeichenkette}
\label{sec:StrCod}
	
	Bei einer Codierung der Daten in einer Zeichenkette werden die Chromosomen als Buchstaben codiert. Dies eignet sich daher entsprechend vor allem für text\-ba\-sier\-te Anwendungen. Aber auch das \textbf{DNA Sequencing Problem} lässt sich damit einfach bearbeiten. Dabei werden die einzelnen DNA-Proteine als Buchstaben bezeichnet, wodurch sich eine Zeichenkette ergibt, die mithilfe von ent\-sprech\-end\-en COs einfach bearbeitet werden kann. \cite{Survey}
	
	Bei einer Zeichenketten-Darstellung der Daten muss man zwischen einer festen Länge der Zeichenkette und einer variablen Länge unterscheiden. Abhängig ist dies vor allem von der Länge der beiden Eltern, also ob die Zeichenketten der Eltern die gleiche oder eine unterschiedliche Länge besitzen.
	
	\subsubsection{Zeichenketten mit festen Längen}
	
		gelten als einfacher zu Handhaben, weshalb diese Art der Repräsentation deutlich häufiger genutzt wird. Grund hierfür liegt darin, dass nicht entschieden werden muss, welche Gene hinzugefügt oder entfernt werden müssen. Jackson hat in seiner Masterarbeit \cite{COProteins} folgende beiden COs vorgestellt, welche man speziell dafür nutzen kann:
		
		\begin{description}
			\item[C1 Crossover] Ermittelt Sekundärstruktur-Elemente und legt die Schnittpunkte auf unstrukturierte Regionen und trennt somit unterschiedliche Strukturtypen.
			\item[C2 Crossover] Legt die Schnittpunkte basierend auf der Frequenz der angrenzenden Genompaare der Sekundärstruktur fest.
		\end{description}
	
	\subsubsection{Zeichenketten mit variablen Längen}
	
		werden dagegen häufig vor allem im Bereich der biologischen Anwendungen wie die Bearbeitung von DNA- bzw. RNA-Sequenzen genutzt. Dies liegt darin Begründet, dass sich diese Art der Darstellung auch in der Natur wiederfindet, wo Chromosomen ebenfalls unterschiedliche Längen haben. Bei Zeichenketten mit variablen Längen liegt eines der Hauptprobleme in der Entscheidung, welche Gene hinzugefügt und welche Gene entfernt werden sollen, wobei die Gesamtanzahl der Gene der Kinder der Anzahl aller Elterngene entsprechen soll. Dazu gibt es \zB folgende COs, welche hierfür genutzt werden können:
		
		\begin{description}
			\item[Internal Crossover] Basiert auf der Annahme, dass Gene der kleineren Sequenz ohne Überhang mit jedem Block der längeren Sequenz angeordnet werden können. \cite{UnequalCO}
			\item[Equal Crossover] Erzeugt Kinder, welche die gleiche Längen wie die Eltern besitzen. \cite{UnqeualLengthCO}
			\item[Inside Crossover] Erzeugt Kinder, die länger als der kurze und kürzer als der lange Elternteil sind. \cite{UnqeualLengthCO}
		\end{description}