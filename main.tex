\documentclass{llncs}
\usepackage{epsfig,amsmath,graphicx,cite,lmodern}
\usepackage{subfig}
\usepackage[utf8]{inputenc}

\begin{document}
\mainmatter

\title{Eine Übersicht über Crossover-Operationen für genetische Algorithmen\\Seminar Organic Computing}
\titlerunning{Eine Übersicht über Crossover-Operationen für genetische Algorithmen}  % abbreviated title (for running head)

\author{Gerald Siegert\\Matrikelnummer: 1450117}
\authorrunning{Siegert} % abbreviated author list (for running head)
\tocauthor{}

\institute{Universität Augsburg\\Lehrstuhl für Organic Computing\\
\email{student@organic-computing.org}}

\maketitle


%-------------------------
\begin{abstract}
	Zusammenfassung des Inhalts des Papers in ca. 200 Wörtern.
\end{abstract}

\pagebreak

\section{Einführung in genetische Algorithmen}

Einführung in genetische Algorithmen, wie funktionieren die Überhaupt

\section{Klassifizierungen von Crossover-Operationen}

Klassifizierung von Crossover

\section{Eindimensionale Repräsentation}

Eindimensionales

\subsection{Binäre Codierung}

Binär

\subsection{Codierung als Ganzzahlen}

Integer

\subsubsection{Operationen für ganzzahlige Werte}

Integer, die nicht als Binär gehandhabt werden

\subsubsection{Operationen für Permutationen}

Permutationen von Integer-Werten (zB TSP)

\subsection{Codierung als Fließkommazahl}

Fließkommazahlen

\subsection{Codierung als Zeichenkette}

String-Codierungen

\section{Mehrdimensionale Repräsentation}

Mehrdimensionale

\subsection{Codierung als Baum}

Bäume und deren nutzen

\subsection{Codierung als Array}

Array und deren Nutzen

\subsection{Weitere Codierungen für mehrdimensionale Daten}

Kurz weiteres wie Matrizen und modularisierte Codierung

\section{Anwendungsspezifische Codierung der Daten}

Kurz anwendungsspezifisches

\section{Universale Crossover-Operationen}

Kurz auf weitere, universal einsetzbare Operationen eingehen (besser am Anfang?)

\section{Zusammenfassung und Ausblick}

Kurze Zusammenfassung

\bibliographystyle{splncs03} 
\bibliography{literature}





\newpage

%-------------------------
\section{Motivation}
\label{sec:Motivation}
Einführung ins Thema. Was bestehen für Probleme, wie soll das gelöst werden? \\

Wieso braucht man vorgestellte Technik/System/Algorithmus?



%-------------------------
\section{Stand der Technik} \label{sec:sdt}
%
Wie andere Verfahren das Problem zu lösen versuchen. \cite{OCBible}

%-------------------------
\section{Hauptteil}
\label{sec:Hauptteil}

\subsection{Grundlagen}
\label{sec:Grundlagen}
Text zu Fig.~\ref{fig:abb1}. Siehe Formel~\ref{eq:mean}

\subsubsection{Advantages and Challenges}
Eine Subsubsection.

\begin{figure}[th!]
	\centering
	\includegraphics[width=.8\columnwidth]{./Figures/Organic-Computing.png}
	\caption{Ein Beispielbild.}
	\label{fig:abb1}
\end{figure}


\begin{figure}
	\centering
	\subfloat[Beispielbild 1]{\includegraphics[width=0.5\columnwidth]{./Figures/Organic-Computing.png}}
	\hfill
	\subfloat[Beispielbild 2]{\includegraphics[width=0.5\columnwidth]{./Figures/Organic-Computing.png}}
	\caption{Zwei Beispielbilder.}
\end{figure}

\begin{equation}
\label{eq:mean}
\bar{e} = \frac{1}{N}\sum_{i=1}^N{\lvert f_i - x_i\rvert}
\end{equation}


%-------------------------
\section{Evaluation}



\end{document}
