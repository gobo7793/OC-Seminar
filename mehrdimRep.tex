% !TEX root = main.tex

\section{Mehrdimensionale Repräsentation}
\label{sec:MehrdimRep}

	Neben einer eindimensionalen Repräsentation der Daten gibt es natürlich auch mehrdimensionale Repräsentationen. Damit sind nicht einfache lineare, sondern komplexere Datenstrukturen gemeint, die entsprechend komplizierter zu Handhaben sind. Da es aber auch Anwendungen gibt, die mehrdimensionale Daten verarbeiten, gibt es auch entsprechende COs, um solche Daten einfach verarbeiten zu können.
	
	\subsubsection{Arrays}
	\label{sec:Arrays}
	
		dürften wohl eine der einfachsten Arten der Mehrdimensionalen Re\-prä\-sen\-ta\-ti\-on\-en darstellen. Sie bilden nicht mehr als eine Verkettung von mehreren linearen Datensätzen. Diese Art der Codierung kann für verschiedene Probleme genutzt werden wie \zB das TSP oder die Suche nach Pareto-Optimaler Lösungen, bei denen nicht alle Zieleigenschaften optimal sein können.
		
		Die dazu passenden COs heißen \uA \textit{Orthogonal Array Based Crossovers}, von denen es zwei Versionen gibt. Die erste Version heißt \textit{Orthogonal Crossover} (CO) und erzeugt die Kinder, indem die Elterngene basierend auf den Kombinationen eines orthogonalen Array ausgewählt werden. Die zweite Version namens \textit{Main Effect Crossover} (MC) wählt die Kindergene im orthogonalen Array dagegen basierend auf deren Effekte aus, wodurch Kinder mit den besten Haupteffekten der Eltern generiert werden. \cite{ArrayCrossover}
	
	\subsubsection{Weitere mehrdimensionale Repräsentationen}
	\label{sec:WeitereMehrdim}
	
		sind beispielsweise \textit{Bäume} oder \textit{Matrizen}, bei denen die Daten entsprechend strukturiert und aufgebaut sind. Vor allem Bäume können bei der Bearbeitung von mehrdimensionalen Daten sehr effizient sein. Beide Darstellungsarten lassen sich \uA auch durch ähnliche COs bearbeiten, wie \zB der \textit{Reijmers Crossover Operator}, welcher aus den Distanzen zwischen den einzelnen Baumknoten eine Distanzen-Matrix erstellt, und diese mit einer vom GA generierten \textit{Correction-Matrix} addiert. \cite{MatrixCO}