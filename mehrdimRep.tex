% !TEX root = main.tex

\section{Mehrdimensionale Repräsentation}
\label{sec:MehrdimRep}

Neben einer eindimensionalen Repräsentation der Daten gibt es natürlich auch mehrdimensionale Repräsentationen. Damit sind nicht einfache lineare, sondern komplexere Datenstrukturen gemeint, die entsprechend komplizierter zu Handhaben sind. Da es aber auch Anwendungen gibt, die mehrdimensionale Daten verarbeiten, gibt es auch entsprechende COs, um solche Daten einfach verarbeiten zu können.

Zunächst wird im Abschnitt \ref{sec:BaumCod} die Codierung der Daten als Baum vorgestellt, im anschließenden Abschnitt \ref{sec:ArrayCod} eine Codierung als Array. Abschließend gibt es im Abschnitt \ref{sec:WeitereMehrdim} noch eine kurze Übersicht über weitere mögliche mehrdimensionale Repräsentationen.

\subsection{Codierung als Baum}
\label{sec:BaumCod}

	Bäume und deren nutzen

\subsection{Codierung als Array}
\label{sec:ArrayCod}

	Array und deren Nutzen

\subsection{Weitere Codierungen für mehrdimensionale Daten}
\label{sec:WeitereMehrdim}

	Kurz weiteres wie Matrizen und modularisierte Codierung
